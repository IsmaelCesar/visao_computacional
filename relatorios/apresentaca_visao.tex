\documentclass[12pt]{beamer}

\usepackage[utf8]{inputenc}
%\usepackage{hyperref}
\usepackage[sort,compress]{cite}
\usepackage{algorithmic}
\usepackage[ruled,linesnumbered,lined]{algorithm2e}
\usepackage{booktabs}

\usepackage{amsfonts}%Usando fontes para conjuntos numéricos

%\usepackage{titlesec}
\usepackage{color,colortbl,multirow}
\usepackage{xy}
\usepackage{graphicx,url, amsmath, color}
\usepackage{graphicx}

\title{Projeto Segunda VA Visão Computacional\\
	Réplica do trabalho:\textbf{Simple face-detection algorithm based on minimum facial features}}
\author{\textbf{Aluno}: Ismael Cesar\\
		\textbf{Professor}: João Paulo}
\date{}
\begin{document}
\frame{
\begin{center}
	\maketitle
 
\end{center}
}

\frame{
	\frametitle{Introdução}
	\begin{itemize}
		\item Detecção de faces pode ser útil em várias aplicações dos dias atuais
		\item Tarefa de detectar pode ser muito custosa
		\item Procurar pelo mínimo de características faciais
		\begin{itemize}
			\item Pele
			\item Cabelo
		\end{itemize}
		\item Deixar detecção de face mais eficiente
	\end{itemize}
}

\frame{
	\frametitle{Conceitos Básicos}
	
	
}

\end{document}
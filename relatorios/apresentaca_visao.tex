\documentclass[12pt]{beamer}

\usepackage[utf8]{inputenc}
%\usepackage{hyperref}
\usepackage[sort,compress]{cite}
\usepackage{algorithmic}
\usepackage[ruled,linesnumbered,lined]{algorithm2e}
\usepackage{booktabs}

\usepackage{amsfonts}%Usando fontes para conjuntos numéricos

%\usepackage{titlesec}
\usepackage{color,colortbl,multirow}
\usepackage{xy}
\usepackage{graphicx,url, amsmath, color}
\usepackage{graphicx}

\title{Projeto Segunda VA Visão Computacional\\
	Réplica do trabalho:\textbf{Simple face-detection algorithm based on minimum facial features}}
\author{\textbf{Aluno}: Ismael Cesar\\
		\textbf{Professor}: João Paulo}
\date{}
\begin{document}
\frame{
\begin{center}
	\maketitle
 
\end{center}
}

\frame{
	\frametitle{Introdução}
	\begin{itemize}
		\item Detecção de faces pode ser útil em várias aplicações dos dias atuais
		\item Tarefa de detectar pode ser muito custosa
		\item Procurar pelo mínimo de características faciais
		\begin{itemize}
			\item Pele
			\item Cabelo
		\end{itemize}
		\item Deixar detecção de face mais eficiente
	\end{itemize}
}

\frame{
	\frametitle{Conceitos Básicos}
	
	\begin{itemize}
		\item Uso de primitivas para computação de valores
		\item Modelo de cor RGB normalizado:	
		\begin{equation}
		\begin{split}
			r  = \frac{R}{R+G+B+\varepsilon} \\
			g  = \frac{G}{R+G+B+\varepsilon} \\
			b  = \frac{B}{R+G+B+\varepsilon}
		\end{split}
		\end{equation}	
	\end{itemize}			
}

\frame{
	\frametitle{Conceitos Básicos}
	
	\begin{itemize}
		\item Primitivas que definem o intervalos de cores para o canal $r$
		\begin{equation}
		\begin{split}
			F_1(r)  = -1.367r^2 + 1.0743r + 0.2 \\
			F_2(r)  = -0.776r^2 + 0.5601r + 0.18
		\end{split}
		\end{equation}
		\item Primitiva para computação dos ton de branco nos canais $r$ e $g$
		\begin{equation}
			White(r,g) = (r - 0.33)^2 + (g - 0.33)^2 
		\end{equation}
	\end{itemize}
}

\frame{
	\frametitle{Conceitos Básicos}
	\begin{itemize}
		\item Primitivas para relações entre o modelo de cor RGB e HSI
		$$
		\theta (R,G,B) = cos^{-1}\left( \frac{0.5((R-G)+(R-B))}{\sqrt{(R-G)^2 +(R-B)(G-R) }} \right)
		$$
		\\
		$$
	Hue(B,G,\theta) = 	\begin{cases}
						\theta,  \text{if } B \leq G \\
						360^\circ - \theta,  \text{if } B > G
						\end{cases}
		$$
		\\
		$$
		I(R,G,B) = 	\frac{1}{3} (R+G+B)
		$$
	\end{itemize}
}

\end{document}